The material in this thesis has not been submitted for examination at the University of Oxford or at any other university. All the work within this thesis is my own, along with all figures (unless explicitly stated otherwise). Work from this thesis has been published in the following works (of which I was first author) in chronological order:

\begin{centering}
\begin{enumerate}
    \item S. Spencer et al., Prospects for the Use of Photosensor Timing Information with Machine Learning Techniques in Background Rejection, PoS(ICRC2019) 798 (2019).
    \item S. Spencer et al., Deep Learning with Photosensor Timing Information as a Background Rejection Method for the Cherenkov Telescope Array, Astroparticle Physics 129C 102579 (2021).
    \item S. Spencer et al., Advanced Studies of SSTCAM Night Sky Background, Spencer et al., SSTCAM Internal Note (2021). 
\end{enumerate}
\end{centering}
Both Publications 1 and 2 were subject to full CTA collaboration review, and Publication 2 also went through a peer review process with Astroparticle Physics. Chapters 3 and 5 detail work that was performed in the context of the CHEC (later SSTCAM) collaborations, and also the wider Analysis and Simulations Working Group (ASWG) of CTA. Chapter 4 contains work performed in collaboration with the VERITAS collaboration.

T. Armstrong performed simulations needed for the results in Chapter \ref{ch:3-TimingInfo}, though the analysis of those simulations was performed by myself. R. Prado played a similar role for the simulations used in Chapter \ref{ch:4-VERITASRealData}, and G. Maier made the changes needed to \textit{Eventdisplay} to make it compatible with the deep learning analysis, although I was responsible for the creation of the rest of the analysis pipeline.