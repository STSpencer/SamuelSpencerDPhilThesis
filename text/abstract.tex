Imaging Atmospheric Cherenkov Telescope arrays allow us to probe the $\gamma$-ray sky from tens of GeV up to hundreds of TeV. They operate by stereoscopically imaging the Cherenkov light generated when an astrophysical $\gamma$-ray interacts with Earth's atmosphere. In order to reject charged cosmic ray events, and to reconstruct the direction and energy of the incident $\gamma$-ray, machine learning methods are used in combination with parametric descriptions of the detected images. One potential means of improving performance for the next-generation Cherenkov Telescope Array (CTA) is to apply new deep learning methods in place of these parametric techniques. In this thesis, we explore the complexity of deploying deep learning methods, first considering the application of high precision timing data, and then testing such methods' performance on real observations from the current generation VERITAS array. Finally, we explore improvements to the modelling of Night Sky Background observed by Cherenkov instruments, that can be used to both inform the design of the Small Sized Telescope Camera (SSTCAM) for CTA, as well as potentially augment simulated data for deep-learning-based event classification.