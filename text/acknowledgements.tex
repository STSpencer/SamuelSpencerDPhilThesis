\subsection*{Personal}

Firstly, I want to thank my supervisor Garret Cotter for getting me through my DPhil, advice writing this thesis and lots of other support over the past four years (including getting me lots of GPUs). Thank you Garret for helping me learn how to be an independent researcher, as well as teaching me how to manage the politics of academia. Similarly, I'd like to thank Gernot Maier for fulfilling a similar role during my time at DESY and since then, working with whom I've learnt a vast amount about $\gamma$-ray data analysis.

Secondly, I want to thank my family. Particularly my parents David and Vivien, as well as my brother Jonathan (and his partner Rebecca), for their support during my DPhil (and beforehand). Completing a doctorate is a challenge at the best of times, but without their help during a global pandemic it would have been impossible.

I particularly want to thank Tom Armstrong, Jason Watson and Rich White for their help during my involvement with CHEC (and now SSTCAM). I'd also like to thank the other members of the $\gamma$-ray astronomy group during my time in Oxford: Paul Morris, Mario H\"{o}rbe, Will Potter, Jamie Davies, Hugh Spackman, Connor Duffy and Jeff Grube for being good travel companions and for the entertaining discussions about $\gamma$-ray astronomy. I’m also thankful to Tarek Hassan, Raul Prado and Mireia Nievas Rosillo for being reliable postdocs during my secondment to DESY. I also owe a (life-long) debt of gratitude to (now all professors) Paula Chadwick, Anthony Brown and Sam Nolan for getting me into the $\gamma$-ray astronomy field in the first place, by hosting me for multiple summer-studentships during my time as an undergraduate in Durham. Thanks as well to all the people who provided comments on the draft of this thesis, particularly Simon Mason.

I’m grateful to Ashling Gordon and Leanne O’Donnell for saving me from admin nightmares multiple times (as well as support during the darkest days of COVID-19). I'm also grateful to Jonathan Patterson for IT support during the GPU upgrade of the \textit{Glamdring} cluster. 

Finally, there’s a long list of friends and colleagues who I’ve met during my time at Oxford Astrophysics (particularly those in the same DPhil year as me), at DESY, and within the CTA consortium, that I don’t have space to thank individually here. I’m very grateful for your help and to have known you all nonetheless.

\newpage

\subsection*{Institutional}
\textbf{All Chapters}: I acknowledge an STFC doctoral studentship, as well as additional support from Oxford University Department of Physics following the COVID-19 pandemic. I also thank the developers of \textit{ctapipe} for providing an invaluable new tool for $\gamma$-ray data analysis. \newline
\textbf{Chapter 3}: This work used the European Grid Infrastructure. Thanks to Luisa Arrabito and Johan Bregeon for their assistance in running simulations and code on it. This work has gone through internal review by the CTA Consortium, who I thank for allowing me to use CTA simulation models. 
\newline
\textbf{Chapters 3 and 4}: This work used the \textit{Glamdring} cluster in Oxford; I thank Jonathan Patterson for his work in upgrading and maintaining this facility, as well as setting up the MongoDB database used for the parallelised optimisation in Chapter \ref{ch:4-VERITASRealData}. 
\newline\textbf{Chapters 3 and 5}:
I gratefully acknowledge financial support from the agencies and organisations listed here: \url{http://www.cta-observatory.org/consortium\_acknowledgments}.
\newline
\textbf{Chapter 4}; I thank the VERITAS Consortium for allowing us to use VERITAS simulation models and data. I also thank the computing support group at DESY for providing cluster resources. \newline
\textbf{Chapter 5}: I thank Matthias Buechele and colleagues in H.E.S.S. for the original development work on the \textit{nsb} tool. This work has made use of data from the European Space Agency (ESA) mission \href{https://www.cosmos.esa.int/gaia}{\textit{Gaia}}, processed by the \textit{Gaia} Data Processing and Analysis Consortium (\href{https://www.cosmos.esa.int/web/gaia/dpac/consortium}{DPAC}). Funding for the DPAC has been provided by national institutions, in particular the institutions participating in the \textit{Gaia} Multilateral Agreement.