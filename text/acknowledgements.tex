\subsection*{Personal}

This is where you thank your advisor, colleagues, and family and friends.

Figures are my own, unless expressly stated otherwise. 


\subsection*{Institutional}

Chapter 3: This work used the European Grid Infrastructure. We thank Luisa Arrabito and Johan Bregeon for their assistance in running our simulations and code on it. Later training runs in this paper used the \textit{Glamdring} cluster in Oxford; we thank Jonathan Patterson for his work in upgrading and maintaining this facility. SS acknowledges an STFC Ph.D. studentship. GC, JW and TA acknowledge support from STFC grants ST/S002618/1 and ST/M00757X/1. GC acknowledges support from Exeter College, Oxford. This work has gone through internal review by the CTA Consortium, who we thank for allowing us to use CTA simulation models. We gratefully acknowledge financial support from the agencies and organisations listed here: \url{http://www.cta-observatory.org/consortium\_acknowledgments}.

Chapter 5:
This work has made use of data from the European Space Agency (ESA) mission \href{https://www.cosmos.esa.int/gaia}{\textit{Gaia}}, processed by the \textit{Gaia} Data Processing and Analysis Consortium (\href{https://www.cosmos.esa.int/web/gaia/dpac/consortium}{DPAC}). Funding for the DPAC has been provided by national institutions, in particular the institutions participating in the \textit{Gaia} Multilateral Agreement. This research made use of Photutils, an Astropy package for detection and photometry of astronomical sources (Bradley et al.
2021). This research made use of Astropy,\footnote{http://www.astropy.org} a community-developed core Python package for Astronomy \cite{astropy:2013},\cite{astropy:2018}.
