\chapter{\label{ch:5-CHECNSB} Advanced Study of SSTCAM NSB}
\minitoc
\section{Introduction}
\label{sec:intro}

\subsection{Scope \& Purpose of this Chapter}
\label{sec:intro:scope}
As we've seen in Chapters \ref{ch:3-TimingInfo} and \ref{ch:4-VERITASRealData}, it is clear that the current modelling of NSB in both the simulation packages \textit{CARE} (used by VERITAS and in Chapter \ref{ch:4-VERITASRealData}) and \textit{sim\_telarray} (used by CTA and H.E.S.S. and in Chapter \ref{ch:3-TimingInfo}) is insufficient to allow for reliable deep-learning-based event classification. In this chapter, we estimate the expected NSB photon rates of SSTCAM under various observing conditions. This modelling will aid SSTCAM design, as well as begin a possible route to improvement in our training data that might one day make deep learning analyses feasible. These investigations have been the subject of an SSTCAM internal note. We perform this by using and expanding upon the capabilities of the \textit{nsb} tool \cite{nsb}, already validated and used by H.E.S.S.. \footnote{For the purposes of this chapter, capitalised NSB refers to the physical phenomenon of Night Sky Background, where as italic lower case \textit{nsb} refers to the Python package used to simulate it.}. Full results from this study, along with sandbox scripts to operate \textit{nsb} can be found on the \url{https://github.com/sstcam/sstcam_NSB} Github page. 

\subsection{Context}
\label{sec:intro:context}
The ability for SSTCAM to operate reliably under partially moonlight and other high NSB observing conditions is critical to the transient and multi-messenger/multi-wavelength science goals of CTA, as it significantly increases the possible observing time. As a result, this ability to operate under moonlit conditions results in a CTA system requirement. The increase in potential science return has been judged to outweigh the associated increase in data rate and person-power required to operate the telescopes. The most notable example for this is the detection of prompt TeV emission from GRB 190114C (the first such detection from the ground), which was detected under partial moonlit (6 times nominal NSB) conditions with MAGIC \cite{magicGRB}. 

Studies of partial moonlight and high NSB observations can also help inform temperature management and SiPM selection for SSTCAM, as well as inform estimations of astrophysical systematic errors. The continuous rate of illumination, analogous to a constant NSB value, across the camera plane affects almost all measurable properties of SiPMs. This is as the bias voltage across the SiPMs is reduced which in turn reduces their sensitivity to light \cite{1mhighnsb}. The majority of previous studies concerning NSB for SiPM-based Cherenkov cameras have primarily focused on NSB's effect on single SiPM pixels \cite{2msipm} \cite{1mcalib} \cite{1mhighnsb} \cite{lstnsb}. The work presented in this chapter is the first detailed simulations of an entire SiPM-based camera's worth of NSB, allowing us to evaluate both the spatial and temporal properties of NSB across the camera plane.

SiPMs are typically operated with a bias circuit with a resistor connected in series, to protect the SiPMs from drawing too much current from the power supply and being damaged. We need to determine the correct bias voltage to set for this system, so we define the overvoltage as $\Delta V=V_{PS}-V_{BD}$, where $V_{BD}$ is the breakdown voltage (the minimum (reverse) bias voltage under which electron avalanche production is possible in an SiPM) and $V_{PS}$ is the bias voltage supplied by the power supply \cite{1mhighnsb}. As a change in NSB rate results in a change of the overvoltage of the SiPMs, which in turn affects the photomultiplier calibration. SSTCAM hopes to counter this with the LED flasher system, and so the rate of change of NSB in the camera affects the number and rate of LED flashes required. We consider the necessary rate of LED flashes to correct for the effects of NSB later in this chapter.
\subsection{Model}
\label{sec:intro:model}

The standard CORSIKA/\textit{sim\_telarray} simulation package \cite{simtel} used for CTA Monte-Carlo simulation productions supports NSB maps as an input parameter, and has some limited functionality to replicate illumination at an infinite distance from the camera (i.e. stars). For reasons of computational efficiency, it does not calculate where in the camera and how bright those stars might be, nor does it realistically simulate the effect of moonlight. We model NSB using the \textit{nsb} package, which assumes that NSB comes from two primary sources. The first is starlight using \textit{Gaia} data, and the second is sky brightness taking account of moonlight and a local extinction coefficient following the semi-analytic model of Krisciunas and Schaeffer \cite{Krisciunas}. The Krisciunas and Schaeffer model was originally fit to data from Mauna Kea before being extended later with additional weight parameters accounting for the relative brightness between the sky and the extra component of starlight in the \textit{hess\_basic} model included in \textit{nsb}. The \textit{hess\_basic} model is expressed as:
\begin{equation}
    f(\rho, A, B, C) = 10^{A}\times (1.06+\cos^2(\rho))+10^{B-\rho/40}+C \times 10^7 \times \rho^2,
\end{equation}

\begin{equation}
    I_{M} (\alpha_{M}) = 10^{-0.4 \times (3.84 + 0.026 \times |\alpha_{M}| + 4 \times 10^{-9} \times \alpha_{M}^4))},
\end{equation}

\begin{equation}
    X(Z)=\begin{cases} (1-0.96\sin^2 Z)^{-0.5} & \mbox{, if }  Z<=\pi/2 \\ (1 - 0.96 \times 1)^{-0.5} & \mbox{, otherwise } \end{cases},
\end{equation}

\begin{equation}
\begin{aligned}
    B_{Moon}(\rho,Z,Z_{M},k,\alpha_{M},A,B,C)=f(\rho,A,B,C) \times I_{M}(\alpha_{M}) \\ \times 10^{-0.4kX(Z_{M})}\\ \times [1-10^{-0.4 k X(Z)}]
\end{aligned}
\end{equation}

\begin{equation}
    B_{Sky}(Z,\phi,B_0,k)=\begin{cases} B_0\times X(Z) \times 10^{-0.4k(X(Z)-1)} & \mbox{, if }  Z<=\pi/2 \\ 0 & \mbox{, otherwise } \end{cases},
\end{equation}

\begin{equation}
    B_{Total}=B_{Moon}+B_{Sky}+B_{Gaia}
\end{equation}

where $B_{Total}$ is the total brightness for a given point on the sky, $B_{Moon}$ is the surface brightness from moonlight, $B_{Sky}$ is the intrinsic surface brightness of the sky, $B_{Gaia}$ is the brightness of stars in the associated healpix pixel \cite{healpix}, $Z$ is the Zenith angle, $\phi$ is the azimuth (with subscript $_M$ referring to Moon parameters rather than source parameters), $f(\rho,A,B,C)$ is the scattering function as a function of lunar great circle separation angle $\rho$ (which takes into account in this form both Rayleigh scattering from atmospheric gases and Mie scattering from atmospheric aerosols), $\alpha_M$ is the phase angle of the Moon and $I_m$ is the illuminance of the Moon in footcandels. $A$,$B$,$C$,$k_0$ and $B_0$ are all free parameters to be determined, values used are in Table \ref{tab:params}. $B_{Total}$ (along with all the other native outputs from the \textit{hess\_basic} \textit{nsb} model) is expressed here in nanoLamberts (nLb), where a brightness in nanoLamberts $B$ relates to magnitudes per square arc second in the V band ($V$) through
\begin{equation}
    \frac{B}{\textrm{nanoLamberts}}=34.08\exp(20.7233-0.92104\times V)
\end{equation}
\cite{Krisciunas}.

The stellar data in \textit{nsb} is obtained using \textit{dr1b} stellar magnitude data from the \textit{Gaia} space mission \cite{gaiadr1a}\cite{gaiadr1b}. Understanding the behaviour of the SiPMs under the illumination of a particularly bright star is important for the future operation plan of SSTCAM. Current practice of most previous Cherenkov cameras is turning the high voltage of blocks of pixels containing such bright stars off. SSTCAM will not have to do this as saturating SiPMs is more tolerable than with them than conventional photomultipliers. If we choose to alter bias voltage for such pixels, it may be adjusted pre-emptively or as a reaction; this is further complicated by the need to use such stars for the slow signal pointing calibration simultaneously. 

Given lack of data from an SSTCAM prototype on the Paranal site it is currently not possible to fit the relative coefficients of the stellar and semi-analytic components of the model. Given the coefficients from HESS and Mauna Kea were within 10\%, we can accept the HESS values as placeholders in the expectation that, if anything, Cerro Paranal will offer superior observing conditions. It should also be noted that the \textit{nsb} package does not explicitly take into account certain other known sources of NSB, in particular contributions from zodiacal light, light from population centres (particularly from mines near $\sim$ 25 km from the Cerro Paranal site \cite{cpmines}), stray reflections from the ground, light from satellites (the \textit{Starlink} satellite is of particular concern, with the satellites reaching $\mathrm{m_G=2.4}$ during the deployment phase, though this decreases to $\mathrm{m_G=6.6}$ afterwards \cite{starlink}) and air glow emission lines in the atmosphere (see Figure \ref{fig:BandE} \cite{BandE}, though these are smaller effects compared to the differences between full moonlight illumination and astronomical dark time. A rough calculation using the spectrum from \cite{BandE} shows that air glow emission lines contribute to a constant NSB value of around 17MHz by themselves in the SSTCAM wavelength range, primarily from the OI 5577\r{A} and Na D 5890\r{A} lines (utilising the standard \cite{BandE} measurements from La Palma). Also not considered here is the potential effect of the 8 magnitude 6.8 laser guide stars used by the E-ELT, though this issue could be potentially mitigated through co-ordinated scheduling \cite{gauglasers}. 
\begin{figure}[ht]
\begin{centering}
%L, B, R, T
\includegraphics[width=\columnwidth]{./figures/bandeplot.png}
\caption{Spectrum of NSB over La Palma on a moonless night, data taken from \cite{BandE} and used in the sim\_telarray NSB rate calculations (but not \textit{nsb}). Note the prominent airglow emission lines at longer wavelengths. Bar the two major emission lines in the V band ($\sim$ 500-700nm), this shows that our approximation of a uniform NSB spectrum in the sensitivity region of SSTCAM is reasonable.}
\label{fig:BandE}
\end{centering}
\end{figure}

\section{Methods}


\subsection{Recommended Parameters}
\label{sec:examples:params}

Many free \textit{nsb} parameters can be set automatically using automatic config generator functions we developed for this project, but some parameters still meed to be determined. Part of the work for this project was determining what to set these values to for simulations of SSTCAM. Recommended \textit{nsb} run parameters are detailed for SSTCAM in Table \ref{tab:params}. We ensure that there is more than one \textit{nsb} FoV map pixel, and more than one healpix pixel within in the diameter of every SSTCAM pixel on sky. Otherwise the \textit{photutils} component of the scripts will fail, nor will it be possible to obtain meaningful Nyquist sampled per-pixel estimates. The parameter selection needed to fulfill these conditions significantly slows \textit{nsb}, but time can be saved by using the same parameter configuration consistently as \textit{Gaia} catalogues and healpix maps are generated whenever the configuration changes. An \textit{nsb} simulation run with these parameters takes approximately 8 hours on a single core to generate an All-Sky Map, an FoV map and a corresponding FITS file (which is typically 763 Mb large using the recommended parameters). This is somewhat longer than the amount of time for which such a map prediction remains accurate, which is around 20 minutes of observing time. It should also be noted that \textit{nsb} is comparatively RAM heavy, needing around 9GB for these conditions. These RAM constraints rule out performing such simulations on the European Grid Infrastructure using CTA-Dirac as typical grid nodes tend to only have 4GB of RAM. Obtaining per-pixel estimates from these FoV maps is much faster (taking only a few minutes).

\begin{table}
    \centering
    \resizebox{\textwidth}{!}{
    \begin{tabular}{c|c|c|c}
         \textbf{Parameter}&\textbf{Value for SSTCAM} & \textbf{Unit}&\textbf{Comments}\\
         \hline
         Observation Latitude & -70.317876 & $^{\circ}$ & Location of Paranal Prod5 SST-1\\
         & (14.974609) & ($^{\circ}$)& (Alternative Value for ASTRI site)\\
         Observation Longitude & -24.681546 & $^{\circ}$ & Location of Paranal Prod5 SST-1 \\
         & (37.693267) & ($^{\circ}$)& (Alternative Value for ASTRI site)\\
         Observation Altitude & 2161.25 & m & Location of Paranal Prod5 SST-1\\
         & (1750) & (m)& (Alternative Value for ASTRI site)\\
         No. Pixels for All Sky Maps & 2000 & Pixels &\\
         Healpix Level & 11 & - & Healpix Cell Needs to be $<$1 Pixel on Sky\\
         \textit{Gaia} Catalog Mag Limit & 15 & mag &  \\
         Moon Above Horizon & 0 & $^{\circ}$ Altitude& Moonlight Observations Limit\\
         Sun Below Horizon & -18 & $^{\circ}$ Altitude & Daylight Threshold\\
         Source Above Horizon & 10 & $^{\circ}$ Altitude & Source Observing Threshold\\
         FoV & 10 & $^{\circ}$ Diameter & Affects WCS Projection\\
         No. Pixels for FoV Map & 10000 & Pixels &  Needs to be $<$1 Pixel on Sky\\
         Sky Pixel Radius & 0.19 & $^{\circ}$ & Pixel Size on Sky\\
         Degree of Gaussian Blurring & 3 & Pixels & \\
         Pixel Shape & Rectangular & - & Pixel Shape on Sky\\
         A & 5.1276 & - & \textit{hess\_basic} Model Parameter \\
         B & 5.9596 & - & \textit{hess\_basic} Model Parameter \\
         C & 0 & - & \textit{hess\_basic} Model Parameter \\
         $B_0$ & 52 & - & \textit{hess\_basic} Model Parameter \\
         $k$ & 0.479 & - & \textit{hess\_basic} Model Parameter \\
         Mirror Area & 7.3 & $m^2$ & Needed for Hz Conversion\\
         PDE & 40 & \% & Needed for Hz Conversion\\
         Telescope Transmission &0.85 & - & Needed for Hz Conversion\\
         Timing Resolution & 15 & Minutes & Needed for Timespan \\&&&and Observation Time Gain Plots\\ 
    \end{tabular}
    }
    \caption{Parameters recommended for use with \textit{nsb} and associated scripts in \textit{sstcam\_nsb}. Note that a small amount of Gaussian blurring is necessary to correct for an apparent bug in \textit{nsb}.}
    \label{tab:params}
\end{table}



\subsection{World Co-ordinate System Description}

To perform aperture photometry on a fits file using co-ordinate data from \textit{ctapipe} \cite{ctapipe2}, one needs to define a World Co-ordinate System (WCS) for the FITS output from \textit{nsb}. This is not automatically written by \textit{nsb} upon the FoV map being created, but it can be provided by the user manually in an \textit{astropy} WCS dictionary. The requisite WCS coefficient values can be determined by dividing the FoV set in nsb by the number of pixels in the FoV map, and setting the centre of the FITS image to the RA/DEC to which \textit{nsb} has been aimed. These are listed in Table \ref{tab:WCS}, assuming that the parameters in Table \ref{tab:params} are used. It should be noted that the CTYPE parameters affect the size of the FoV that needs to be used with \textit{nsb}, tangential projections (TAN) are preferable as they result in only a 10 degree FoV being required to reliably perform that aperture photometry, which reduces the computational cost of running \textit{nsb}, an alternative healpix projection requires 12 degrees to be simulated to prevent Not a Number (NaN) values being produced at the camera edge by photutils.
\begin{table}[]
    \centering
    \resizebox{0.8\textwidth}{!}{

    \begin{tabular}{c|c|c}
         \textbf{WCS Header Name}&\textbf{Value for SSTCAM} & \textbf{Comments}\\
         \hline
         CTYPE1 & `RA---TAN' & X-Axis Projection\\CUNIT1&`deg'& X-Axis Unit\\
         CDELT1&0.001 &X-Axis Increment\\
         CRPIX1&5000&X-Axis Reference Point\\
         CRVAL1&Determined by Pointing&X-Axis Reference Value\\
         NAXIS1&10000&No. Pixels X-Axis\\
         CTYPE2&`DEC--TAN'&Y-Axis Projection\\
         CUNIT2&`deg'&Y-Axis Unit\\
         CDELT2&0.001&Y-Axis Increment\\
         CRPIX2&5000&Y-Axis Reference\\
         CRVAL2&Determined by Pointing&Y-Axis Reference Value\\
         NAXIS2&10000&No. Pixels Y-Axis\\
         CROTA1&0&Field Rotation\\
         CROTA2&0& Field Rotation\\
         RADESYS&`ICRS'& Co-Ordinate System\\

    \end{tabular}
    }
    \caption{WCS Dictionary Headers and Parameters Recommended for SSTCAM, Reliant on Configuration in Table \ref{tab:params}. Note the CTYPEn projection used affects the nessecary size of the FoV map generated by \textit{nsb}. The rotation parameters CROTA1 and CROTA2 are degenerate in their effect, and are not trivially set.}
    \label{tab:WCS}
\end{table}

\subsection{Aperture Photometry}
To accurately determine the NSB rates in individual pixels, one must perform aperture photometry on the FoV maps generated by \textit{nsb}. This can be performed using the pixel position information from \textit{ctapipe} \cite{ctapipe2} for a given observation time, location, altitude and azimuth. We then use the \textit{photutils} astropy-affiliated \cite{astropy:2018} photometry package \cite{photutils} to integrate the FoV map around the pixel positions, assuming they have a rectangular geometry on-sky and a known pixel diameter (0.19 degrees). In effect, this numerically performs the integral
\begin{equation}
    B_{Pixel}=\int B_{Total} \cdot dA
\end{equation}
where $B_{Pixel}$ is the brightness of a camera pixel and $dA$ is a rectangular surface element on the sky, so it should be noted that $B_{Pixel}$ is therefore not a linear function of $B_{Moon}$, $B_{Sky}$ or $B_{Gaia}$. Validation was performed to ensure that the rotation of the camera images generated were consistent with the \textit{ctapipe} Engineering Camera Frame.

This produces pixel maps in units of nLb/pixel, but in practice we need to convert this to Hz/pixel (which is the directly observable quantity). To do this, we assume the spectrum from \textit{nsb} to be uniform and perform the following calculation. Assuming that the results from \textit{nsb} covers the BP spectrum, and that they are spread evenly across the band, this implies

\begin{equation}
    \mathrm{B (photons/(ns\ sr\ nm\ m^2) )=\frac{B (Lamberts)}{(10^4 / \pi \times E \times (680-330) \times 10^9)}}
\end{equation}

where $\mathrm{E=hc/505nm}$, we then multiply this by the range 300-550 nm, multiply by an assumed 40\% Photo Detection Efficiency (PDE), and then multiply by the solid angle subtended by a pixel ($8\times 10^{-6}$ sr), the mirror area ($7.3 \mathrm{m^2}$) and the telescope transmission ($0.85$) to get an NSB rate in Hz. A similar result can be obtained through scaling the spectrum from Benn and Ellison \cite{BandE} (seen in Figure \ref{fig:BandE}) and using the existing \textit{sstcam-simulation} package \cite{sstcamsimulation}, however the presence of strong emission lines in that spectrum results makes that method less appropriate, as it is unlikely that light from stars or moonlight will follow such a strongly peaked spectrum. The results from this analysis are written to an \textit{astropy} table whereby they can be further analysed at will. It should be noted that the workflow developed for this work could be reasonably trivially used by all CTA cameras, provided these basic parameters are known.

Throughout this work we assume a CTA CHEC-S pixel geometry is roughly indicative of the final SSTCAM, though this will need to be updated once a full Monte Carlo model of the engineering camera has been created and a decision on the SiPMs for SSTCAM has been made. The value most likely to affect the results presented here is the angular extent of the pixels on the sky. We also use the `SST-ASTRI' optical structure as defined by \textit{ctapipe}, though the effect of this upon the analysis is minimal, as even the effective focal length of the telescope optics is set manually to 2.15191 m. Another likely source of complication that is not modelled here (as there's no trivial means of modelling it without resorting to full ray-tracing), is that for a 2 mirror Schwartzchild-Couder optical design telescope (with which SSTCAM was designed to operate) moonlight can reflect of the secondary mirror in different ways compared to a single mirror design.

\label{sec:examples}
\subsection{Pointing Validation}
To verify the pointing accuracy of the \textit{nsb} fields, we ran an observation of Polaris through the \textit{nsb} chain (from the ASTRI site on Sicily), and verified the position of Polaris in the frame using the `V/50' Vizier catalogue \cite{vizier}. This is somewhat complicated by the fact that the WCS rotation angle keywords CROTA1 and CROTA2 are free parameters (partly because the \textit{ctapipe} skycoord.rotation and skycoord.roll fields are presently unfilled in the latest version of \textit{ctapipe}), but this investigation demonstrates that at least the source pointed at will be within the FoV of the camera.
\begin{figure}[ht!]
\begin{centering}
%L, B, R, T
\resizebox{0.7\columnwidth}{!}{\includegraphics[trim=0cm 0cm 0cm 0cm,clip=true,width=\columnwidth]{./figures/polaris.png}}
\caption{Per pixel NSB values for Polaris (position circled in the Engineering Camera Frame), note that these runs used a lower than normal magnitude limit level (8) for speed, which causes the artefacts in the image. The expected position of Polaris is circled.}
\label{fig:polaris}
\end{centering}
\end{figure} 

\section{Results and Discussion}
\subsection{Eta Carinae Observations Under Varying Conditions}
\label{sec:etacarvary}
As a reasonable worst-case scenario, we consider four observing scenarios of the colliding wind binary Eta Carinae. Eta Carinae has recently been the subject of a major paper by H.E.S.S. \cite{hessetacar}, and it is basically the hardest known gamma-ray source to observe with IACTs given the high stellar density in the region and the high number of UV photons produced (which causes false triggers, though this is not taken into account by \textit{nsb}). Firstly we consider a 'dark' field, at the same altitude as Eta Carinae but differing azimuth, for comparison. This appears to be a reasonable choice of dark field more generally, as there is only a single star with a V-band magnitude brighter than 3 (Zeta Centauri, $m_V=2.55$), though through a search of NASA's HEASARC catalogue there are 8 4FGL \textit{Fermi} LAT point sources in the dark field, which may potentially have spectra that extend into the SST energy range (although this is unlikely).  Secondly we consider observing Eta Carinae during astronomical dark time (at the same observing time as the dark field). Thirdly, we observe the Eta Carinae region with half moonlight present. This half moonlit scenario is designed to partially replicate the conditions for CTA requirement B-SST-1680, the moon is above the horizon and has 0.53 Fractional Lunar Illumination (FLI). Finally we observe the region under full moonlight. The full moon scenario represents the worst possible observing conditions for Eta Carinae (and by extension the worst observing conditions possible), with the moon both at 1.0 FLI and well above the horizon. The significant all-sky illumination caused by this full moon, which completely dominates over the background stars, can be seen in Figure \ref{fig:allskyetacarextreme}.

In addition to pixel-wise analysis, we can also bin mean NSB values per superpixel and TM. This is important for understanding the triggering process, which relies on charge values per superpixel rather than per pixel, and for thermal control of the TM electronics. These results can be seen in Figures \ref{fig:etacarpixel},\ref{fig:etacarsuperpixel} and \ref{fig:etacarTM}.

\begin{table}[h]
    \centering
    \resizebox{\textwidth}{!}{

    \begin{tabular}{c|c|c|c|c|c}
         \textbf{Observation}&\textbf{Observing Time}&\textbf{ALT ($^{\circ}$)}&\textbf{AZ ($^{\circ}$)} & \textbf{RA ($^{\circ}$)}&\textbf{DEC ($^{\circ}$)} \\ 
         \hline
         Dark 'Empty' Field & 2022-02-07T04:54:0 & 33.1 & 126.7 & 162.0 & -43.0\\
         Eta Carinae No Moonlight &2022-02-07T04:54:0 & 33.1 & 146.7 &161.3&-59.7\\
         Eta Carinae Half Moonlight &20-02-21T01:10:0 & 12.5 & 150.8 &161.3&-59.7\\
         Eta Carinae Full Moonlight &2022-05-16T07:00:00& 13.1&209.6&161.3&-59.7
         
    \end{tabular}
    }
    \caption{Observation Parameters for the Four Eta Carinae Runs.  The observing Altitude (ALT) and Azimuth (AZ) are presented, along with the simulated source Right Ascension (RA) and Declination (DEC). The moonlit Eta Carinae runs are at a low altitude that an IACT would not normally observe at, but since the \textit{nsb} model does not take into account atmospheric properties this is not consequential. Times are in UTC.}
    \label{tab:etacar_params}
\end{table}

\begin{figure}[ht]
\begin{centering}
%L, B, R, T
\includegraphics[width=0.8\columnwidth]{./figures/allskyetacarextreme.png}
\caption{An all sky plot for the full moonlight run, highlighting the presence and brightness of the moon for this observation.}
\label{fig:allskyetacarextreme}
\end{centering}
\end{figure}

%% Figure example 
\begin{figure}[h]
%L, B, R, T
\begin{minipage}{\linewidth}\centering
\subcaptionbox{Dark Empty Field}{\includegraphics[width=0.48\linewidth]{./figures/etacardarkgauss3_Hz_pixel.png}}
\subcaptionbox{No Moonlight Eta Carinae}{\includegraphics[width=0.48\linewidth]{./figures/etacarbrightgauss3_Hz_pixel.png}}
\subcaptionbox{Half Moonlight Eta Carinae}{\includegraphics[width=0.48\linewidth]{./figures/etacarmoonpoint53gauss3_Hz_pixel.png}}
\subcaptionbox{Full Moonlight Eta Carinae}{\includegraphics[width=0.48\columnwidth]{./figures/etacarnightmarev2_Hz_pixel.png}}
\caption{Per Pixel NSB Values for Eta Carinae Under the Four Observing Scenarios Described.}
\label{fig:etacarpixel}

\end{minipage}
\end{figure} 

\begin{figure}[h]
%L, B, R, T
\begin{minipage}{\linewidth}\centering
\subcaptionbox{Dark Empty Field}{\includegraphics[width=0.48\linewidth]{./figures/Hz_Superpixel_dark.png}}
\subcaptionbox{No Moonlight Eta Carinae}{\includegraphics[width=0.48\linewidth]{./figures/Hz_Superpixel_bright.png}}
\subcaptionbox{Half Moonlight Eta Carinae}{\includegraphics[width=0.48\linewidth]{./figures/Hz_Superpixel_moonpoint53.png}}
\subcaptionbox{Full Moonlight Eta Carinae}{\includegraphics[width=0.48\columnwidth]{./figures/Hz_Superpixel_nightmare.png}}
\caption{Mean NSB Values Per Superpixel for Eta Carinae Under the Four Observing Scenarios Described.}
\label{fig:etacarsuperpixel}
\end{minipage}
\end{figure} 

\begin{figure}[h]
%L, B, R, T
\begin{minipage}{\linewidth}\centering
\subcaptionbox{Dark Empty Field}{\includegraphics[width=0.48\linewidth]{./figures/etacardarkgauss3_Hz_TM.png}}
\subcaptionbox{No Moonlight Eta Carinae}{\includegraphics[width=0.48\linewidth]{./figures/etacarbrightgauss3_Hz_TM.png}}
\subcaptionbox{Half Moonlight Eta Carinae}{\includegraphics[width=0.48\linewidth]{./figures/etacarmoonpoint53gauss3_Hz_TM.png}}
\subcaptionbox{Full Moonlight Eta Carinae}{\includegraphics[width=0.48\columnwidth]{./figures/etacarnightmarev2_Hz_TM.png}}
\caption{Mean NSB Values Per TM for Eta Carinae Under the Four Observing Scenarios Described.}
\label{fig:etacarTM}
\end{minipage}
\end{figure} 


\subsection{Changes Between Adjacent Pixels for Eta Carinae}
\label{sec:etacarajacent}

We also analyse the same data for the purposes of determining the maximum change between horizontally and vertically adjacent pixels, superpixels and Target modules.
We did this by devising a function that, for every pixel in a 2D array, obtains the maximum change in mean NSB rate between itself and its horizontally and vertically adjacent neighbours (if they exist), then assigns that value to the pixel. This calculation negates changes in the PSF across the focal plane in the camera. The results from this investigation are shown in Figures \ref{fig:diffetacarpixel}, \ref{fig:diffetacarsuperpixel} and \ref{fig:diffetacarTM}.



\begin{figure}[ht]
%L, B, R, T
\begin{minipage}{\linewidth}\centering
\subcaptionbox{Dark Empty Field}{\includegraphics[width=0.48\linewidth]{./figures/Hz_pixel_diff_dark.png}}
\subcaptionbox{No Moonlight Eta Carinae}{\includegraphics[width=0.48\linewidth]{./figures/Hz_pixel_diff_bright.png}}
\subcaptionbox{Half Moonlight Eta Carinae}{\includegraphics[width=0.48\linewidth]{./figures/Hz_pixel_diff_moonpoint53.png}}
\subcaptionbox{Full Moonlight Eta Carinae}{\includegraphics[width=0.48\columnwidth]{./figures/Hz_pixel_diff_nightmare.png}}
\caption{Maximum Differences Between Adjacent Pixels in NSB Values for Eta Carinae Under the Four Observing Scenarios Described.}
\label{fig:diffetacarpixel}

\end{minipage}
\end{figure} 

\begin{figure}[h]
%L, B, R, T
\begin{minipage}{\linewidth}\centering
\subcaptionbox{Dark Empty Field}{\includegraphics[width=0.48\linewidth]{./figures/Hz_Superpixel_diff_dark.png}}
\subcaptionbox{No Moonlight Eta Carinae}{\includegraphics[width=0.48\linewidth]{./figures/Hz_Superpixel_diff_bright.png}}
\subcaptionbox{Half Moonlight Eta Carinae}{\includegraphics[width=0.48\linewidth]{./figures/Hz_Superpixel_diff_moonpoint53.png}}
\subcaptionbox{Full Moonlight Eta Carinae}{\includegraphics[width=0.48\columnwidth]{./figures/Hz_Superpixel_diff_nightmare.png}}
\caption{Difference in Mean NSB Values Between Adjacent Superpixels for Eta Carinae Under the Four Observing Scenarios Described.}
\label{fig:diffetacarsuperpixel}

\end{minipage}
\end{figure} 

\begin{figure}[h]
%L, B, R, T
\begin{minipage}{\linewidth}\centering
\subcaptionbox{Dark Empty Field}{\includegraphics[width=0.48\linewidth]{./figures/Hz_TM_diff_dark.png}}
\subcaptionbox{No Moonlight Eta Carinae}{\includegraphics[width=0.48\linewidth]{./figures/Hz_TM_diff_bright.png}}
\subcaptionbox{Half Moonlight Eta Carinae}{\includegraphics[width=0.48\linewidth]{./figures/Hz_TM_diff_moonpoint53.png}}
\subcaptionbox{Full Moonlight Eta Carinae}{\includegraphics[width=0.48\columnwidth]{./figures/Hz_TM_diff_nightmare.png}}
\caption{Difference in Mean NSB Values Between Adjacent TMs for Eta Carinae Under the Four Observing Scenarios Described.}
\label{fig:diffetacarTM}

\end{minipage}
\end{figure}

In the case of individual pixels, it seems like the pixels adjacent to pixels with high NSB from stars are largely isolated from other potential sources of NSB, however once these results are binned and averaged by superpixel and TM that one stars to frequently see the effect of multiple stars overlapping. From these results, it appears clear that the regions of the camera that will need to be most robust to significant routine changes in NSB are the pixels, superpixels and TMs nearest to SSTCAM's uninstrumented corners, but more generally pixels, superpixels and TMs need to be robust to a single pixel being at an illumination of 1GHz whilst its neighbours are illuminated at a few hundred MHz.

\subsection{Change in Eta Carinae NSB Over Time}
To determine the maximum possible change in NSB through the camera over a run, we ran \textit{nsb} a half hour before each of the observations in Subsection \ref{tab:etacar_params} (tracking the source). This choice was based on the fact that the source sets after the full moonlight run, and also removes potential issues related to normalisation that we discuss in the next subsection. Whilst these results in Figures \ref{fig:30minsetacarpixel}, \ref{fig:30minsetacarsuperpixel} and \ref{fig:30minsetacarTM} show that the brightness of an individual pixel can change on the order of a GHz per 30 minutes, changes in the mean illumination of the camera are much slower, being of the order of 2MHz/hour, with the exception of the fully moonlit scenario, where the mean change over half an hour is dominated by a bright star.



\begin{figure}[ht]
%L, B, R, T
\begin{minipage}{\linewidth}\centering
\subcaptionbox{Dark Empty Field}{\includegraphics[width=0.48\linewidth]{./figures/diff_Hz_pixel_dark.png}}
\subcaptionbox{No Moonlight Eta Carinae}{\includegraphics[width=0.48\linewidth]{./figures/diff_Hz_pixel_bright.png}}
\subcaptionbox{Half Moonlight Eta Carinae}{\includegraphics[width=0.48\linewidth]{./figures/diff_Hz_pixel_moonpoint53.png}}
\subcaptionbox{Full Moonlight Eta Carinae}{\includegraphics[width=0.48\columnwidth]{./figures/diff_Hz_pixel_nightmare.png}}
\caption{Changes in Eta Carinae NSB values Per Pixel Over 30 Minutes}
\label{fig:30minsetacarpixel}

\end{minipage}
\end{figure} 

\begin{figure}[ht]
%L, B, R, T
\begin{minipage}{\linewidth}\centering
\subcaptionbox{Dark Empty Field}{\includegraphics[width=0.48\linewidth]{./figures/diff_Hz_Superpixel_dark.png}}
\subcaptionbox{No Moonlight Eta Carinae}{\includegraphics[width=0.48\linewidth]{./figures/diff_Hz_Superpixel_bright.png}}
\subcaptionbox{Half Moonlight Eta Carinae}{\includegraphics[width=0.48\linewidth]{./figures/diff_Hz_Superpixel_moonpoint53.png}}
\subcaptionbox{Full Moonlight Eta Carinae}{\includegraphics[width=0.48\columnwidth]{./figures/diff_Hz_Superpixel_nightmare.png}}
\caption{Changes in Eta Carinae NSB values Per Superpixel Over 30 Minutes}
\label{fig:30minsetacarsuperpixel}

\end{minipage}
\end{figure} 

\begin{figure}[ht]
%L, B, R, T
\begin{minipage}{\linewidth}\centering
\subcaptionbox{Dark Empty Field}{\includegraphics[width=0.48\linewidth]{./figures/diff_Hz_TM_dark.png}}
\subcaptionbox{No Moonlight Eta Carinae}{\includegraphics[width=0.48\linewidth]{./figures/diff_Hz_TM_bright.png}}
\subcaptionbox{Half Moonlight Eta Carinae}{\includegraphics[width=0.48\linewidth]{./figures/diff_Hz_TM_moonpoint53.png}}
\subcaptionbox{Full Moonlight Eta Carinae}{\includegraphics[width=0.48\columnwidth]{./figures/diff_Hz_TM_nightmare.png}}
\caption{Changes in Eta Carinae NSB values Per TMs Over 30 Minutes}
\label{fig:30minsetacarTM}
\end{minipage}
\end{figure} 
A procedure to deal with the moon entering the FoV (potentially by accident) has not yet been established for SSTCAM (one option is to pre-emptively drop the gain for the entire camera). However, attempting to run NSB for an observation whereby the moon is directly in the field crashes \textit{nsb}. As such, the change in the full moonlit Eta Carinae field represents the greatest shift in average NSB rate that can be simulated (30Hz/Minute/pixel) with \textit{nsb}.

\subsection{Eta Carinae NSB Per Pixel}
\label{sec:etacartimespan}
\begin{figure}[ht]
\begin{centering}
%L, B, R, T
\resizebox{\columnwidth}{!}{\includegraphics[width=\columnwidth]{./figures/etacartimespan.png}}
\caption{Pixel brightness in Hz for Eta Carinae over a year, normalised to a mean observing rate of 76Hz.}
\label{fig:etacar_timespan}
\end{centering}
\end{figure}

To determine the limit of brightness in a single pixel over a year, \textit{nsb} has a tool to perform this. However, said tool only provides results in nLb at a particular infinitesimal point, which is something of a meaningless quantity. As such, for running this tool for SSTCAM with NSB, we assume our non-moonlight Eta Carinae NSB run from the previous run represents a typical observation, and scale the mean of the year long timespan plot to the mean pixel value for that run. In the case of the maximum NSB variation over a single night, we instead normalise the maximum of the NSB rate to the maximum of the year long plot (to which the single night's observation corresponds).

Figure \ref{fig:etacar_timespan} clearly shows that whilst there is modulation in the apparent brightness of the region given lunar phase and separation, there is also a very strong dependence on moon altitude, with the brightest times being when the moon is both high on the sky, near 1.0 FLI, and close to the source. Throughout these results and those in subsection \ref{sec:etacarvary} it appears as though lunar illumination, separation and altitude will be the dominant cause of both high NSB and rapid changes in it.

There is some inconsistency between this result and the result for the entire camera on the same night with a 30 minute difference, which shows an overall drop in NSB over the same time window, although both show extreme variation. This could be the result of a minor bug in \textit{nsb}, or it could be a results of the infinitesimally small region of the sky corresponding to Eta Carinae brightening whilst the larger FoV is darkening. However, this suggests that even on the most extreme sources changes in NSB rate per pixel should be within the capacity of the LED flasher system to correct. 

\begin{figure}[ht]
\begin{centering}
%L, B, R, T
\resizebox{\columnwidth}{!}{\includegraphics[trim=0cm 0cm 0cm 0cm,clip=true,width=\columnwidth]{./figures/timespan_Hz_etacarmaxnorm.png}}
\caption{Pixel brightness in Hz for Eta Carinae for the brightest observing night in 2022, normalised to a 264.8 Hz max observation rate (as determined from Figure \ref{fig:etacar_timespan}).}
\label{fig:etacar_timespan_extreme}
\end{centering}
\end{figure}
\newpage
\subsection{Observing Time Gains}
\textit{nsb} similarly has a function to provide plots of the observation time gained as a function of NSB threshold, however this again is in nLb at an infinitesimal point. As such, we treat our Eta Carinae dark frame as nominal, and create a timespan plot for that night and observing location. The total possible observing time is obtained by summing all the elements in the brightness array where the brightness is less than a given threshold , and then multiplying this by the time resolution (as the time resolution for this brightness array is a pre-set parameter).
\begin{figure}[ht]
\begin{centering}
%L, B, R, T
\resizebox{\columnwidth}{!}{\includegraphics[trim=0cm 0cm 0cm 0cm,clip=true,width=\columnwidth]{./figures/timespan_dark.png}}
\caption{NSB timespan plot for the dark field used to calibrate the observing time gain calculation }
\label{fig:timespan_dark}
\end{centering}
\end{figure}

We then divide the observation time gain plots by a the nominal NSB value obtained from this measurement (50 nLb), and perform an observing time gain calculation for the Vela Pulsar (Figure \ref{fig:obstimegainvelapulsar})and the Blazar Markarian 421 (Figure \ref{fig:obstimegainmrk421}, in order to consider differences from galactic vs extragalactic sources) from the Paranal site, considering a year at the Paranal site starting at 2022-01-27.

\begin{figure}[h!]
\begin{centering}
%L, B, R, T
\includegraphics[width=\columnwidth]{./figures/obstime_VelaPulsar_nom.png}
\caption{Observing time gain for the Vela Pulsar as a function of nominal NSB.}
\label{fig:obstimegainvelapulsar}
\end{centering}
\end{figure}

\begin{figure}[h!]
\begin{centering}
%L, B, R, T
\includegraphics[width=\columnwidth]{./figures/obstime_mrk421_nom.png}
\caption{Observing time gain for Markarian 421 as a function of nominal NSB.}
\label{fig:obstimegainmrk421}
\end{centering}
\end{figure}

Whilst the potential observation time gains for the Vela Pulsar are significant, the potential benefits of operating at a higher NSB rate are more muted for Markarian 421. This is partly due to the location of the Paranal site, and as such Markarian 421 is typically lower on the sky, but it is also likely that the fact that Markarian 421 is away from the galactic plane (with fewer stars in the field) contributes to this. This therefore bodes well for potential SSTCAM galactic transient science, such as the recurrent nova RS Ophiuchi (recently detected by H.E.S.S. in ATEL 14857).

\subsection{Exact Matching of Half-Moonlight Requirement}
Whilst a reasonable approximation given the complexity of observing a particular source, the Half-Moonlight Eta Carinae runs are not an exact match to the CTA SST Requirement B-SST-1680, which states
\begin{centering}
    SSTs must be capable of gamma-ray observations with uniform night sky background illumination levels up to at least $\mathrm{4.3\ photons\ ns^{-1} sr^{-1} cm^{-2}}$ in the wavelength range 300-650 nm with the Moonlight Reference Spectrum.
\end{centering}
this also requires that the moon have 0.5 FLI and is at an altitude of 45 degrees and the telescope is pointed at the zenith. To compare to our half-moonlit Eta Carinae run, we ran \textit{nsb} runs with exactly these parameters for UTC time 2022-03-24T07:32:00.000.

\begin{figure}[ht]
%L, B, R, T
\begin{minipage}{\linewidth}\centering
\subcaptionbox{NSB per pixel.}{\includegraphics[width=0.48\linewidth]{./figures/hm_pixel.png}}
\subcaptionbox{NSB per superpixel.}{\includegraphics[width=0.48\linewidth]{./figures/hm_Superpixel.png}}
\subcaptionbox{NSB per TMs.}{\includegraphics[width=0.48\linewidth]{./figures/hm_TM.png}}

\caption{Results for half-moonlit zenith observations.}

\end{minipage}
\label{fig:hm}
\end{figure}
This result demonstrates that if SSTCAM performance requirements up to a 160MHz NSB rate can be met, the overall camera will function well under most realistic observing scenarios. The question becomes if individual pixels will be able to handle such NSB rates without an overwhelming gain$\times$PDE$\times$OCT drop.

These results are slightly more pessimistic than the half-moonlit Eta-Carinae runs, but are still within a factor of around 2 of the same values, and are largely consistent with values generated using sim\_telarray (which for Prod3b gets between 160 and 200MHz,    depending on silicon selection, based on scaling the Benn and Ellison spectrum under the same conditions, albeit with a 2 telescope multiplicity cut we can't replicate here). It should be noted that the existing half-moonlight investigation for the SST-1M project \cite{1mcalib} estimated a value of 670MHz for the same conditions, other than human error a possible source of the difference is the physical size of the photomultiplier pixels, which for SSTCAM is 6/7mm ($0.19^{\circ}$ on sky) as opposed to around 2.32cm ($0.24^{\circ}$ on sky) for Digicam (the SST-1M prototype camera), however other differences between the photomultipliers used (such as PDE) might also affect the results. 

\subsection{Required Flasher Timespans}

The angular field rotation of a star $R$ can be described as a function of Latitude (Lat), Altitude (Alt) and Azimuth (Az) using the formula
\begin{equation}
    R(Lat,Alt,Az)=K\cos(Lat)\frac{\cos(Az)}{\cos(Alt)}
    \label{eq:rot}
\end{equation}
where the constant $K$ is 15.04 degrees/hour.

Using this formula, it is possible to extract the number of required flasher calibration pulses to reach a mean flasher error level. If we assume a mean illumination level from the flasher of 50 pe and that we can only extract the charge to 40\% worse than poisson, then we need 40 seconds (i.e. 400 pulses) to reach 1\% error on the mean flasher level per pixel

Figure \ref{fig:rot40s} shows the areas on the sky where a 1\% flasher error can be achieved given this model of angular field rotation. The sky positions (assuming the Paranal site) of bright stars (brighter than 4 mag) from the Hipparcos catalogue are calculated using the \textit{Skyfield} package and are also plotted.
\begin{figure}[h]
\begin{centering}
%L, B, R, T
\includegraphics[width=0.7\columnwidth]{./figures/rot40s.png}
\caption{Field rotation with flasher calibration limit, assuming 40s flasher calibration duration.}
\label{fig:rot40s}
\end{centering}
\end{figure}

This clearly causes a slight error in calibration along the North-South axis, but this is tolerable given the rate at which stars move across the sky. By doubling the error budget and reducing the duration of the flashing procedure one can achieve much greater calibrated sky coverage, as seen in Figure \ref{fig:rot10s}, with only a handful of bright stars falling outside the calibrated range.

\begin{figure}[h]
\begin{centering}
%L, B, R, T
\includegraphics[width=0.7\columnwidth]{./figures/rot10s.png}
\caption{Field rotation with flasher calibration limit, assuming 10s flasher calibration duration.}
\label{fig:rot10s}
\end{centering}
\end{figure}

\subsection{Star Tracking Investigations}

To understand the short term effect of stars moving through the field, we ran \textit{nsb} fields at 3 minute intervals, pointing at the zenith under half-moonlit conditions, and observed both the absolute pixel NSB values and the overall change since the first field. Note that this is also a worst case scenario observing the fastest possible changes in stellar position, since $R$ in Equation \ref{eq:rot} is largest for a given observing latitude at the zenith.

Despite the PSF not being particularly well modelled, it appears that bright stars move to adjacent pixels over a roughly 15 minute timespan, with light being partially 'split' between multiple pixels and the associated ratio changing over order 3 minutes.
\begin{figure}[ht]
%L, B, R, T
\begin{minipage}{\linewidth}\centering
\subcaptionbox{Initial Field}{\includegraphics[width=0.48\linewidth]{./figures/Hz_pixel_3m.png}}
\subcaptionbox{Pixel Values After 3 Minutes}{\includegraphics[width=0.48\linewidth]{./figures/Hz_pixel_6m.png}}
\subcaptionbox{Pixel Values After 6 Minutes}{\includegraphics[width=0.48\linewidth]{./figures/Hz_pixel_9m.png}}
\subcaptionbox{Pixel Values After 9 Minutes}{\includegraphics[width=0.48\columnwidth]{./figures/Hz_pixel_12m.png}}
\subcaptionbox{Pixel Values After 12 Minutes}{\includegraphics[width=0.48\columnwidth]{./figures/Hz_pixel_15m.png}}

\caption{Pixel NSB Values in Hz, pointing at the zenith under 'half-moonlit' conditions with high resolution timing, note in particular the changes to the three brightest pixels in the images.}

\end{minipage}
\label{fig:zenithpixel}
\end{figure} 

\begin{figure}[ht]
%L, B, R, T
\begin{minipage}{\linewidth}\centering
\subcaptionbox{Total Change in Pixel Values After 3 Minutes}{\includegraphics[width=0.48\linewidth]{./figures/diff_Hz_pixel_3m.png}}
\subcaptionbox{Total Change in Pixel Values After 6 Minutes}{\includegraphics[width=0.48\linewidth]{./figures/diff_Hz_pixel_6m.png}}
\subcaptionbox{Total Change in Pixel Values After 9 Minutes}{\includegraphics[width=0.48\linewidth]{./figures/diff_Hz_pixel_9m.png}}
\subcaptionbox{Total Change in Pixel Values After 12 Minutes}{\includegraphics[width=0.48\columnwidth]{./figures/diff_Hz_pixel_12m.png}}

\caption{Overall Change in Pixel NSB Values in Hz for the High Time Resolution Star Tracking Investigation.}

\end{minipage}
\label{fig:zenithpixelchange}
\end{figure} 

\subsection{Results for a Magnitude 0 star}
For comparison with other ray tracing methods, we investigate the brightness of Rigel (m=0.12) under astronomical dark time (2021-12-08T04:48:00.000). This shows a magnitude 0 star corresponds to around 900MHz in a single pixel (lower than previous expected), though the movement of the star through multiple pixels in the camera plane could affect this value as seen in the previous subsection.
\begin{figure}[h]
\begin{centering}
%L, B, R, T
\includegraphics[width=0.7\columnwidth]{./figures/Hz_pixel_Rigel.png}
\caption{Rigel observed with \textit{nsb} under astronomical dark time conditions.}
\label{fig:rigel}
\end{centering}
\end{figure}
\section{Conclusions}

In this report, we have examined the potential effects of NSB for SSTCAM. We have presented detailed simulations of the background to Eta Carinae under four observing scenarios, calculated the effect on individual pixels observing the same field over time, calculated the potential observing time to be gained for two sources by operating at a high NSB level, verified the half moonlight CTA requirement, investigated on short timescales the movement of stars through the camera and considered the ability of the LED flasher system to compensate for NSB from stars.

Whilst the \textit{nsb} package has proven capable of generating useful results for SSTCAM, it currently does not meet CTA software requirements for test coverage or code quality (and development by H.E.S.S. on the package has stalled). There would be significant advantages to recreating its functionality in the future with a \textit{ctapipe}-affiliated project (following modern code development principles) for all CTA instruments. Notably \textit{nsb's} speed could likely be improved by using Just-In-Time compilation and a storage medium for healpix data that was quicker to access and parse than a text file (as is currently the case).

It should be noted that the results presented here are a significantly simplified analysis. In particular, we do not consider the effects of OCT, focal plane curvature or non-uniformity of the PSF across the camera plane. Similarly, there is a potential for 'ghost' stars to appear in the field of view as a result of reflections of starlight from the SiPMs or window. We also neglect any wavelength dependence of PDE or telescope transmission. Similarly we also don't take into account lightning, which can potentially cause rapid changes in the illumination of the camera, but this can be reliably countered by using a weather-system based safety. That said, our results are significantly more realistic than those available natively in sim\_telarray, are broadly as expected, and appear to be largely consistent with values obtained from dedicated full Monte Carlo simulations.


%%%%%%%%%%%%%%%%%%%%%%%%%%%%%%
%%%%% Appendices
%%%%%%%%%%%%%%%%%%%%%%%%%%%%%%

