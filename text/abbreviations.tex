% First parameter can be changed eg to "Glossary" or something.
% Second parameter is the max length of bold terms.
\begin{mclistof}{List of Abbreviations}{3.2cm}

\item[1D, 2D, 3D] One, two or three dimensional, referring in this thesis to spatial dimensions in an image.
\item[AI] Artificial Intelligence
\item[AGN] Active Galactic Nuclei
\item[ALT] Altitude
\item[AZ] Azimuth
\item[ASTRI] Astrofisica con Specchi a Tecnologia Replicante Italian;a CTA SST optical structure prototype.
\item[AUC] Area Under Curve Score; a metric used in machine learning.
\item[BDT] Boosted Decision Tree; a machine learning technique for classification and regression.
\item[CHEC] The Compact High Energy Camera; a prototype camera for the SSTs.
\item[CNN] Convolutional Neural Network, a form of machine learning algorithm designed to work with images.
\item[CTA] Cherenkov Telescope Array, the planned successor to the current generation of IACTs.
\item[CTAO] Cherenkov Telescope Array Observatory, the observatory associated with CTA.
\item[DEC] Declination
\item[DL] Deep Learning. Refers to machine learning with a high degree of depth (or equivalently abstraction of learned features).
\item[EAS] Extensive Air Shower; a particle cascade in the atmosphere.
\item[FACT] First G-APD Cherenkov Telescope; An IACT on La Palma.
\item[FLI] Fractional Lunar Illumination; The percentage of the Moon illuminated.
\item[FNR] False Negative Rate
\item[FPR] False Positive Rate
\item[FSRQ] Flat Spectrum Radio Quasar; a class of AGN.
\item[GAN] Generative Adversarial Network; a deep learning method of performing pseudo-simulation and domain adaptation.
\item[GCT] Gamma Cherenkov Telescope;a CTA SST optical structure prototype.
\item[HAWC] High Altitude Water Cherenkov; A water Cherenkov detector in Mexico.
\item[H.E.S.S.] High Energy Stereoscopic System; An IACT array in Namibia.
\item[HV] High Voltage
\item[IACT] Imaging Atmospheric Cherenkov Telescope.
\item[IC] Inverse Compton
\item[ImPACT] Image Pixelwise fit for Atmospheric Cherenkov Telescopes; a template fitting method of event reconstruction used by H.E.S.S..
\item[KSP] Key Science Project; a major CTA scientific investigation.
\item[LHASSO] Large High Altitude AirShower Observatory; a hybrid water Cherenkov/ IACT detector under construction in China.
\item[LST] Large Size Telescope; the largest CTA telescope class designed to operate at the lowest energies.
\item[LSTM] Long Short-Term Memory Network; a machine learning method for handling data series.
\item[MAGIC] Major Atmospheric Gamma Imaging Cherenkov Telescopes; an IACT array on La Palma.
\item[ML] Machine Learning
\item[MST] The Medium Size Telescope; the intermediate size CTA telescope class.
\item[NaN] Not a Number
\item[NSB] Night Sky Background to IACTs
\item[OCT] Optical Cross-Talk; an effect observed in SiPMs.
\item[OSO 3] Orbiting Solar Observer 3
\item[PDE] Photo-Detection Efficiency; the percentage of photons incident on a PMT that are detected.
\item[pe] Photo-electrons
\item[PMT] Photomultiplier Tube; a high-sensitivity light detector.
\item[PSF] Point Spread Function
\item[RA] Right Ascension
\item[RF] Random Forest; a machine learning technique similar to BDTs.
\item[ROC] Receiver Operator Characteristic Curve; A curve to demonstrate machine learning performance. 
\item[SAS-2] Small Astronomy Satellite 2
\item[SCT] The Swartzchild Couder Telescope; an alternative, American lead dual mirror prototype design for the MST.
\item[SiPM] Silicon Photomultiplier; the type of photomultiplier used by CHEC-S
\item[SST] Small Size Telescope; the smallest CTA telescope designed to operate at the highest energies.
\item[SSTCAM] Small Size Telescope Camera; the camera for the SSTs.
\item[TM] Target Module
\item[TNR] True Negative Rate
\item[ToO] Target of Opportunity
\item[TPR] True Positive Rate
\item[VERITAS] Very Energetic Radiation Imaging Telescope Array System; An IACT array in Arizona.
\item[WCS] World Co-ordinate System; a means of defining the location of an image on the sky.
\end{mclistof} 
